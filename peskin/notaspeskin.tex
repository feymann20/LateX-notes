\documentclass{article}
\usepackage{meuestilo}
\title{Notas do Peskin-Grupo de Estudos}
\author{Natã Hora}

\begin{document}
	\maketitle
	\section{Quantização dos campos}
	O inicio do cap 2 do peskin comenta sobre teoria classica de campos porém prefiro a abordagem de outros livros nao irei me preocupar aq em anotar-lo
	Para quantizarmos, promovemos os campos $\phi,\pi$ a operadores e impomos que devem satisfazer relações de comutações, relembrando que para o caso de graus de liberdades finitos temos que
	
	$$ \qty[q_i,p_j]=i \delta_{ij}; \quad \qty[q_i,q_j]=0=\qty[p_i,p_j]$$	


	Para o sistema contínuo a generalização é dada por 
	$$ \qty[\phi(\vb{x}),\pi(\vb{y})]=i\delta^{(3)}(\vb{x-y}); \quad \qty[\phi(\vb{x}),\phi(\vb{y})]=0=\qty[\pi(\vb{x}),\pi(\vb{y})]$$	
	
Considerando que esses operadores não dependem do tempo ( retrato de Schrödinger)

Tendo um campo que satisfaz a equação de KG temos que a sua expansão no espaço de Fourier será dada por ( o fator de normalização terei que estudar mais sobre).

$$\phi ( \vb(x),t)=\int \frac{\dd[4]{p}}{(2\pi)^3}e^{i \vb{p} \vdot \vb{x}} \phi( \vb{p},t)$$

Assim a equação de KG fica 


$$\qty\big[\Box+m^2] \int \frac{\dd[4]{p}}{(2\pi)^3}e^{i \vb{p} \vdot \vb{x}} \phi( \vb{p},t)=$$
$$\Box\int \frac{\dd[4]{p}}{(2\pi)^3}e^{i \vb{p} \vdot \vb{x}} \phi( \vb{p},t)
+m^2\int \frac{\dd[4]{p}}{(2\pi)^3}e^{i \vb{p} \vdot \vb{x}} \phi( \vb{p},t)=$$
$$\qty{\big[\pdv[2]{t}+(|\vb{p}|^2+m^2)]}\phi(\vb{p},t)=0$$

Para essa equação não possuir soluções triviais e se assumirmos que o campo não depende expliticamente do tempo temos que

 $$(-|\vb{p}|^2+m^2)\phi(\vb{p},t)=0$$
 
Logo o  $-|\vb{p}|^2+m ^2=0 \iff E^2$ 








\end{document}