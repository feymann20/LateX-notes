\documentclass{article}
\usepackage{meuestilo}
\title{Mecânica Clássica, Segundo Contato}
\author{H. Natã}
\begin{document}
 \maketitle
 \section{Formulação Newtoniana}
 \subsection{Cinemática}
 
 Dado um referencial, a posição de um ponto pode ser especificado dado pelo vetor posição 
 $$\vb{x}=x_i\vb{e}_i$$
 
 Isso é justamente uma combinação linear dos vetores da base que assumimos aqui sejam ortogonais e normalizados e também estamos utilizando a convenção da Einstein. Os pontos podem ser parametrizados por um parâmetro $t$ que será o nosso tempo,que será assumido monotonicamente crescente deixando evidente a noção de passado e futuro. 
 
 Uma quantidade importante que possuimos na física é a distância. No ponto de vista matemático a distância é conhecida como a métrica do espaço que sendo estudado. Por exemplo, a métrica euclidiana definida por:
 $$d(x,y)=\sqrt{\sum_{i}(x_i-y_i)^2}$$
 
 Essa definição funciona muito bem no espaço físico que por enquanto estamos lidando.Também temos interesse buscar propriedades do vetor posição, como a sua derivada temporal, 
 $$\vb{v}(t)=\dot{\vb{x}}(t)$$
 
 
 Seja s um parametro que varia suavemente e monotonicamente ao longo da trajetória  e seja $\vb{x}(s_0),\vb{x}(s_1)$ pontos nesse trajeto. Assim podemos definir a distancia ao longo do trajeto entre esses dois pontos por
 

$$l(s_0,s_1)= \int_{s_0}^{s_1} \qty\big(\dv{x_i}{s}\dv{x_i}{s})^{1/2}) ds$$
 
 
 Como $l$ foi tomado como parametro a velocidade será dada pela regra da cadeia.
 
 $$\vb{v}=\dv{\vb{x}}{l}\dv{l}{t}$$
 
 A primeira derivada é apenas o vetor $\vb*{\tau}$ tangente à trajetória no tempo $t$. Então $\vb{v}=\vb*{\tau}v$. E assim podemos definir agora a aceleração derivando novamente no tempo, e realizando o processo semelhantmente obtemos que
 
 $$\vb{a}=\dv{\vb*{\tau}}{t}v+\vb*{\tau} \dv{v}{t}$$
 
 Esse vetor está relacionado pela curvatura da trajetória. É fácil de mostrar que $\vb*{\dot{\tau}}$ é perpendicular a curva podemos então definir um novo vetor nessa direção unitário $\vb{n}=\dot{\vb*{\tau}}/|\dot{\vb*{\tau}}|$ E definindo, $\kappa= |\dv{\vb*{\tau}}{l}|$ Temos que a aceleração será dada por
 
 $$\vb{a}=\kappa v^2 \vb{n}+ \ddot{l}\vb*{\tau}$$ E podemos ainda definir o vetor perpendicular a esses dois vetores unitários que é dito vetor binormal $\vb{B}$ definido pelo produto vetorial que sua derivada em relação ao comprimento do arco mede e define a torsão.
\subsection{Principios da dinamica}

Principio 1: Existe um referencial, chamado de inercial que possui as seguintes propriedades: Toda particula isolada se move em uma linha reta nesse referencial ( segue a geodesica em um espaço plano), e A noção de tempo é quantificada de tal forma que a particula isolada se move em uma velocidade constante. Principio 2: Conservação do momentum,



\subsubsection{Consequencias das Equações de Newton}

$$\vb{F(x,\dot{x},t)}=m\dv[2]{\vb{x}}{t}$$

A solução dessa equação diferencial existe e é única,em exceção de casos especiais, dada as condições iniciais. Agora considere outro referencial que o vetor posição é $\vb{y}$. Deve existe uma transformação que relacione esses dois referenciais ( uma transformação afim).

$$y_i=f_i(x,t), \quad x_i=g_i(y,t)$$

Derivando no tempo temos que


$$\dot{y}_i= \pdv{f_i}{x_j}x_j+\pdv{f_i}{t}$$
$$ \ddot{y}_i=\pdv{f_i}{x_j}\dot{x}_j+\pdv[2]{f_i}{x_j}{x_k}\dot{x}_j\dot{x}_j+2\pdv[2]{f_i}{x_j}{t}\dot{x}_j+\pdv[2]{f_i}{t}$$

Não podemos dizer nada sobre os dois sistemas de coordenada, pórem se forem cartesianos, as transformações tem de ser lineares i.e

$$y_i=f_{ik} (t)x_k+b_i(t)$$

Se os referenciais são inerciais eles estao se movendo em velocidade relativa constante isto nos leva ( não tão trivialmente) que $f_{ik}(t)=\phi_{ik},b_i= \beta_i t$ Então a aceleração é dada por:

$$\ddot{y}_i=\phi_{ik}\ddot{x}_k$$

Assim temos as relações entre os referenciais inerciais



Temos agora as seguintes definições: O momento de uma partícula é definido por:

$$\vb{p}=m\vb{v}$$


Momento Angular:

$$\vb{L}= \vb{x} \wedge \vb{p}$$
$$\vb{\dot{L}}=\vb{x} \land \vb{F} \equiv \vb{N}$$


\subsubsection{Energia e Trabalho}

Se a força for uma função apenas do tempo a vetor posição pode ser escrito como

$$\vb{x}=\vb{x}(t_0)+t(t_0)\vb{v}(t_0)+ \frac{1}{m} \int_{t_0}^{t}dt' \int_{t_0}^{t'}\vb{F}(t'')dt''$$

Se a força for apenas função de $\vb{x}$ temos que a integral de linha da força será

$$\int_{\vb{x}(t_0)}^{\vb{x}(t)} \vb{F}(\vb{x}) \vdot \dd{\vb{x}}= \frac{1}{2} m \Delta v^2$$

Se a força for conservativa temos a existencia de uma função potencial $V$ que seu o negativo do seu gradiente é a força que não é unica que difere por uma constante

Assim podemos definir a energia mecânica do sistema físico,
\newpage

\subsection*{Sistema de N particulas}

Seja $\vb{F}_ij$ ,esse elemento é antissimetrico, a força total sobre a partícula é dada por

$$m_i\vb{\ddot{x}}_i=\vb{F}_i+\sum_{j=1}^{N} \vb{F}_ij$$

Poŕem se somarmos em para todas as particulas temos que

$$\sum_{i}m_i \vb{\ddot{x}}=\sum_{i} \vb{F_i}=F$$

E agora podemos redefinir essa equação da seguinte maneira


$$\vb{F}=M\vb{\ddot{X}}$$

O momento é definido de forma similar, e assim obtemos as variaveis dinamicas do centro de massa do sistema.

\section{Formulação Lagrangiana}

Os sistemas físicos agora que iremos estudar são aqueles que estão restritos a uma região espaço esses sistemas são ditos sistemas vinculados, o espaço em que ocorre o movimento essa restrição é dito a variedade de configuração $\mathbb{Q}$. Onde as nossas coordenadas serão chamadas de coordenadas generalizadas $q^\alpha$ e a quantidade delas sera o grau de liberdade do movimento que será também a dimensão de $\mathbb{Q}$.

Supondo que estamos lidando com N partículas e os vínculos são dados por K equações da forma:

$$f_i(\vb{x}, \dots ,\vb{x}_N,t))=0 , i=1 , \dots K <3N$$


As funções f são supostas diferenciáveis nos seus argumentos e o parametro $t$ informa que ela varia independente do movimento das partículas. Os vinculos dados são ditos holonomos que significam integráveis. e Os não holonomos são do tipo

$$f_i(\vb{x} \dots \vb{x}_N,\vb{\dot{x}}_1,\dots\vb{\dot{x}}_N,t)=0 , i=1 , \dots K <3N$$


















\end{document}